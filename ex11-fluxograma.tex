\documentclass{article}
\usepackage{amsmath}
\usepackage{color,pxfonts,fix-cm}
\usepackage{latexsym}
\usepackage[mathletters]{ucs}
\DeclareUnicodeCharacter{46}{\textperiodcentered}
\DeclareUnicodeCharacter{58}{$\colon$}
\DeclareUnicodeCharacter{32}{$\ $}
\usepackage[T1]{fontenc}
\usepackage[utf8x]{inputenc}
\usepackage{pict2e}
\usepackage{wasysym}
\usepackage[english]{babel}
\usepackage{tikz}
\pagestyle{empty}
\usepackage[margin=0in,paperwidth=595pt,paperheight=842pt]{geometry}
\begin{document}
\definecolor{color_283006}{rgb}{1,1,1}
\definecolor{color_29791}{rgb}{0,0,0}
\begin{tikzpicture}[overlay]
\path(0pt,0pt);
\filldraw[color_283006][nonzero rule]
(-15pt, 10.25pt) -- (580.5pt, 10.25pt)
 -- (580.5pt, 10.25pt)
 -- (580.5pt, -832pt)
 -- (580.5pt, -832pt)
 -- (-15pt, -832pt) -- cycle
;
\filldraw[color_283006][nonzero rule]
(-15pt, 10.25pt) -- (580.7245pt, 10.25pt)
 -- (580.7245pt, 10.25pt)
 -- (580.7245pt, -832.3175pt)
 -- (580.7245pt, -832.3175pt)
 -- (-15pt, -832.3175pt) -- cycle
;
\end{tikzpicture}
\begin{picture}(-5,0)(2.5,0)
\put(198.5557,-69.59735){\fontsize{12.00452}{1}\usefont{T1}{cmr}{b}{n}\selectfont\color{color_29791}Fluxograma: nome do fluxograma}
\put(225.6714,-105.611){\fontsize{12.00452}{1}\usefont{T1}{cmr}{b}{n}\selectfont\color{color_29791} Felipe Silva Fernandes}
\put(221.6972,-123.6177){\fontsize{12.00452}{1}\usefont{T1}{cmr}{b}{n}\selectfont\color{color_29791}Leandro Soares Amorim}
\put(213.831,-141.6245){\fontsize{12.00452}{1}\usefont{T1}{cmr}{b}{n}\selectfont\color{color_29791}Luiz Eduardo de Lima Dias}
\put(234.3934,-213.6517){\fontsize{12.00452}{1}\usefont{T1}{cmr}{b}{n}\selectfont\color{color_29791} 27 de abril de 2025}
\put(6.788095,-249.6652){\fontsize{12.00452}{1}\usefont{T1}{cmr}{m}{n}\selectfont\color{color_29791} Resumo Assunto: Jogo Adventure Text.}
\put(6.788095,-267.672){\fontsize{12.00452}{1}\usefont{T1}{cmr}{m}{n}\selectfont\color{color_29791} O programa Adventure Text é um jogo onde o usuário fará escolhas a partir de duas perguntas, se ele}
\put(6.788095,-285.6788){\fontsize{12.00452}{1}\usefont{T1}{cmr}{m}{n}\selectfont\color{color_29791}acertar ele vence o jogo se ele errar perderá o jogo. Neste artigo iremos apresentar o seu fluxograma}
\put(6.788095,-303.6855){\fontsize{12.00452}{1}\usefont{T1}{cmr}{m}{n}\selectfont\color{color_29791}completo. Após a modelagem do fluxograma e desenvolvimento da lógica de programação em algoritmo, o}
\put(6.788095,-321.6924){\fontsize{12.00452}{1}\usefont{T1}{cmr}{m}{n}\selectfont\color{color_29791}programa será implementado na Linguagem de Programação C. }
\put(6.788095,-357.7059){\fontsize{12.00452}{1}\usefont{T1}{cmr}{b}{n}\selectfont\color{color_29791}Local: Escola Politécnica de Pernambuco - UPE/POLI ´}
\put(6.788095,-375.7127){\fontsize{12.00452}{1}\usefont{T1}{cmr}{b}{n}\selectfont\color{color_29791}Orgão Financiador: N/A}
\put(6.788095,-393.7195){\fontsize{12.00452}{1}\usefont{T1}{cmr}{b}{n}\selectfont\color{color_29791}Caracterização: Modelagem, Projeto e Implementação de Software em Linguagem em C.}
\put(6.788095,-429.7331){\fontsize{12.00452}{1}\usefont{T1}{cmr}{b}{n}\selectfont\color{color_29791}Introdução: Este programa é uma implementação de um jogo de aventura interativo baseado em texto, onde}
\put(6.788095,-447.7399){\fontsize{12.00452}{1}\usefont{T1}{cmr}{m}{n}\selectfont\color{color_29791}o jogador é desafiado a tomar decisões em sequência, influenciando o fluxo do jogo e determinando o seu}
\put(6.788095,-465.7466){\fontsize{12.00452}{1}\usefont{T1}{cmr}{m}{n}\selectfont\color{color_29791}resultado final. O jogo inicia com a apresentação de uma história introdutória, fornecendo o contexto da}
\put(6.788095,-483.7534){\fontsize{12.00452}{1}\usefont{T1}{cmr}{m}{n}\selectfont\color{color_29791}aventura. A partir daí, o jogador é solicitado a escolher um objeto de uma lista fornecida. A escolha do}
\put(6.788095,-501.7602){\fontsize{12.00452}{1}\usefont{T1}{cmr}{m}{n}\selectfont\color{color_29791}objeto é crucial, pois o jogo valida se ele é correto. Caso o jogador selecione o objeto errado, o jogo resulta}
\put(6.788095,-519.767){\fontsize{12.00452}{1}\usefont{T1}{cmr}{m}{n}\selectfont\color{color_29791}em uma morte prematura, e o processo é encerrado. Caso contrário, o jogo prossegue para a próxima etapa,}
\put(6.788095,-537.7738){\fontsize{12.00452}{1}\usefont{T1}{cmr}{m}{n}\selectfont\color{color_29791}onde o jogador deve escolher um verbo para interagir com o ambiente de forma apropriada. Assim como na}
\put(6.788095,-555.7806){\fontsize{12.00452}{1}\usefont{T1}{cmr}{m}{n}\selectfont\color{color_29791}etapa anterior, o verbo escolhido é validado, e, caso seja incorreto, o jogador morre, encerrando a partida. Se}
\put(6.788095,-573.7874){\fontsize{12.00452}{1}\usefont{T1}{cmr}{m}{n}\selectfont\color{color_29791}o verbo escolhido for correto, o jogador alcança a vitória e o jogo chega ao seu término bem-sucedido. O}
\put(6.788095,-591.7941){\fontsize{12.00452}{1}\usefont{T1}{cmr}{m}{n}\selectfont\color{color_29791}ciclo de decisões (escolha de objeto e verbo) gera uma dinâmica onde o jogador precisa tomar as decisões}
\put(6.788095,-609.8009){\fontsize{12.00452}{1}\usefont{T1}{cmr}{m}{n}\selectfont\color{color_29791}corretas para evitar falhar e alcançar a conclusão desejada, com um fluxo de execução controlado por}
\put(6.788095,-627.8077){\fontsize{12.00452}{1}\usefont{T1}{cmr}{m}{n}\selectfont\color{color_29791}condições de validação. O programa será modelado em fluxograma em uma primeira fase, em seguida sua}
\put(6.788095,-645.8145){\fontsize{12.00452}{1}\usefont{T1}{cmr}{m}{n}\selectfont\color{color_29791}lógica será desenvolvida em formato de algoritmo, para então na terceira fase será implementado em}
\put(6.788095,-663.8213){\fontsize{12.00452}{1}\usefont{T1}{cmr}{m}{n}\selectfont\color{color_29791}Linguagem de Programação C. }
\end{picture}
\newpage
\begin{tikzpicture}[overlay]
\path(0pt,0pt);
\filldraw[color_283006][nonzero rule]
(-15pt, 10.25pt) -- (580.5pt, 10.25pt)
 -- (580.5pt, 10.25pt)
 -- (580.5pt, -832pt)
 -- (580.5pt, -832pt)
 -- (-15pt, -832pt) -- cycle
;
\end{tikzpicture}
\begin{picture}(-5,0)(2.5,0)
\put(-15,-831.7628){\includegraphics[width=594.9742pt,height=841.8172pt]{latexImage_542529a56d2348b24b9dc169d65429e5.png}}
\end{picture}
\end{document}